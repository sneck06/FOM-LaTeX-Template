\section{Building a language}
For a compiler or an interpreter to be able to interpret a \ac{DSL}, the language must be accurately and precisely defined. Accurately means that the language must be defined consistently down to the smallest detail. Precisely means in this case that all aspects of the language must be laid out. If parts of the language are inconsistent or too vague, authors are forced to interpret these aspects themselves. This inevitably leads to different authors having different approaches to the same problem. If a \ac{DSL} is to be created that meets the criteria described above, two components are needed. The first component is a set of rules, also called syntax. The second component is a formal definition of the meaning, also called semantics. \parencite[cf.][p. 2]{farrell_compiler_1995}
\subsection{Syntax}
The first step when defining syntax is defining an alphabet. This alphabet consists of tokens, which do not necessarily have to be letters. Several tokens, formulated according to a set of rules, make up a sentence or string. The alphabet of the English language is, in the context of syntax, not a list of the permissible characters, which is predominantly called the alphabet or 'ABC', but the permissible tokens.
E.g. in the sentence 'the donkey screams' the tokens 'the', 'donkey' and 'screams' are part of the alphabet of the English language. The token 'gHArFk' consists of permissible characters but is not part of the valid alphabet. However, the use of permissible tokens alone does not make a sentence correct. The sentence 'on sleep blue' consists of tokens that are part of the English alphabet, but it is still not a valid sentence. The correct application of the rule set is still missing, in this example a missing object. Only the correct use of the alphabet AND the set of rules make a sentence syntactically correct. \parencite[cf.][p. 2]{farrell_compiler_1995}\\
If the alphabet and the set of rules are notated in a normal form, they can be called grammar. Relevant to this thesis is the \ac{EBNF}, which will be described in section \ref{sec:EBNF}
\subsection{Extended Backus-Naur Form}\label{sec:EBNF}
\parencite{backus_report_1960} -> \parencite{wirth_what_1977} -> \parencite{isoiec_149771996e_information_1996} -> \parencite{zaytsev_bnf_2012}
