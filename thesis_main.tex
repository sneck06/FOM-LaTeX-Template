%-----------------------------------
% Define document and include general packages
%-----------------------------------
% Tabellen- und Abbildungsverzeichnis stehen normalerweise nicht im
% Inhaltsverzeichnis. Gleiches gilt für das Abkürzungsverzeichnis (siehe unten).
% Manche Dozenten bemängeln das. Die Optionen 'listof=totoc,bibliography=totoc'
% geben das Tabellen- und Abbildungsverzeichnis im Inhaltsverzeichnis (toc=Table
% of Content) aus.
% Da es aber verschiedene Regelungen je nach Dozent geben kann, werden hier
% beide Varianten dargestellt.
\documentclass[12pt,oneside,titlepage,listof=totoc,bibliography=totoc]{scrartcl}
%\documentclass[12pt,oneside,titlepage]{scrartcl}

%-----------------------------------
% Dokumentensprache
%-----------------------------------
%\def\FOMEN{}% Auskommentieren um die Dokumentensprache auf englisch zu ändern
\newif\ifde
\newif\ifen

%-----------------------------------
% Meta information
%-----------------------------------
%-----------------------------------
% Meta Informationen zur Arbeit
%-----------------------------------

% Autor
\newcommand{\myAutor}{Sebastian Bunge}

% Adresse
\newcommand{\myAdresse}{Lindenstra\ss e 15c \\ \> \> \> 53227 Bonn}

% Titel der Arbeit
\newcommand{\myTitel}{Development of a Query Language for Full-Text Search in Relational Databases}

% Betreuer
\newcommand{\myBetreuer}{Prof. Dr. Peter Steininger}

% Lehrveranstaltung
\newcommand{\myLehrveranstaltung}{Modul}

% Matrikelnummer
\newcommand{\myMatrikelNr}{539441}

% Ort
\newcommand{\myOrt}{Bonn}

% Datum der Abgabe
\newcommand{\myAbgabeDatum}{\today}

% Semesterzahl
\newcommand{\mySemesterZahl}{7}

% Name der Hochschule
\newcommand{\myHochschulName}{FOM Hochschule für Oekonomie \& Management}

% Standort der Hochschule
\newcommand{\myHochschulStandort}{Bonn}

% Studiengang
\newcommand{\myStudiengang}{Wirtschaftsinformatik}

% Art der Arbeit
\newcommand{\myThesisArt}{Bachelor Thesis}

% Zu erlangender akademische Grad
\newcommand{\myAkademischerGrad}{Bachelor of Science (B.Sc.)}

% Firma
\newcommand{\myFirma}{Deutsche Telekom IT GmbH}


\ifdefined\FOMEN
    %Englisch
    \entrue
    \usepackage[english]{babel}
\else
    %Deutsch
    \detrue
    \usepackage[ngerman]{babel}
\fi


\newcommand{\langde}[1]{%
   \ifde\selectlanguage{ngerman}#1\fi}
\newcommand{\langen}[1]{%
   \ifen\selectlanguage{english}#1\fi}
\usepackage[utf8]{luainputenc}
\langde{\usepackage[babel,german=quotes]{csquotes}}
\langen{\usepackage[babel,english=british]{csquotes}}
\usepackage[T1]{fontenc}
\usepackage{fancyhdr}
\usepackage{fancybox}
\usepackage[a4paper, left=4cm, right=2cm, top=4cm, bottom=2cm]{geometry}
\usepackage{graphicx}
\usepackage{colortbl}
\usepackage[capposition=top]{floatrow}
\usepackage{array}
\usepackage{float}      %Positionierung von Abb. und Tabellen mit [H] erzwingen
\usepackage{footnote}
% Darstellung der Beschriftung von Tabellen und Abbildungen (Leitfaden S. 44)
% singlelinecheck=false: macht die Caption linksbündig (statt zentriert)
% labelfont auf fett: (Tabelle x.y:, Abbildung: x.y)
% font auf fett: eigentliche Bezeichnung der Abbildung oder Tabelle
% Fettschrift laut Leitfaden 2018 S. 45
\usepackage[singlelinecheck=false, labelfont=bf, font=bf]{caption}
\usepackage{caption}
\usepackage{enumitem}
\usepackage{amssymb}
\usepackage{mathptmx}
%\usepackage{minted} %Kann für schöneres Syntax Highlighting genutzt werden. ACHTUNG: Python muss installiert sein.
\usepackage[scaled=0.9]{helvet} % Behebt, zusammen mit Package courier, pixelige Überschriften. Ist, zusammen mit mathptx, dem times-Package vorzuziehen. Details: https://latex-kurs.de/fragen/schriftarten/Times_New_Roman.html
\usepackage{courier}
\usepackage{amsmath}
\PassOptionsToPackage{table}{xcolor}\usepackage[most]{tcolorbox}
\tcbset{standard jigsaw, opacityback=0, colframe=black, sharp corners}
\usepackage{marvosym}			% Verwendung von Symbolen, z.B. perfektes Eurozeichen
\usepackage[nounderscore]{syntax}	% Write proper Backus-Naur Form, no underscore is needed to avoid problems with the cite function
\setlength{\grammarindent}{0pt}	% No indent in grammar to not having to use paragraphs and waste space

\renewcommand\familydefault{\sfdefault}
\usepackage{ragged2e}

% Mehrere Fussnoten nacheinander mit Komma separiert
\usepackage[hang,multiple]{footmisc}
\setlength{\footnotemargin}{1em}

% todo Aufgaben als Kommentare verfassen für verschiedene Editoren
\usepackage{todonotes}

% Verhindert, dass nur eine Zeile auf der nächsten Seite steht
\setlength{\marginparwidth}{2cm}
\usepackage[all]{nowidow}

%-----------------------------------
% Farbdefinitionen
%-----------------------------------
\definecolor{darkblack}{rgb}{0,0,0}
\definecolor{dunkelgrau}{rgb}{0.8,0.8,0.8}
\definecolor{hellgrau}{rgb}{0.0,0.7,0.99}
\definecolor{mauve}{rgb}{0.58,0,0.82}
\definecolor{dkgreen}{rgb}{0,0.6,0}

%-----------------------------------
% Pakete für Tabellen
%-----------------------------------
\usepackage{epstopdf}
\usepackage{nicefrac} % Brüche
\usepackage{multirow}
\usepackage{rotating} % vertikal schreiben
\usepackage{mdwlist}
\usepackage{tabularx}% für Breitenangabe

%-----------------------------------
% sauber formatierter Quelltext
%-----------------------------------
\usepackage{listings}
\usepackage{listings-rust}

% linenumber fix for listings
\usepackage{etoolbox}% http://ctan.org/pkg/etoolbox
\makeatletter
\patchcmd{\lst@GLI@}% <command>
  {\def\lst@firstline{#1\relax}}% <search>
  {\def\lst@firstline{#1\relax}\def\lst@firstnumber{#1\relax}}% <replace>
\makeatother

\lstdefinelanguage{Fulltext-Search}{
	keywords={@contains, @startswith, @inflection, @thesaurus, @near, @weighted},
	keywordstyle=\color{blue}\bfseries,
	ndkeywords={:, +, &, |, !, -},
	ndkeywordstyle=\color{darkgray}\bfseries,
	stringstyle=\color{red}\ttfamily,
	morestring=[b]"
}

\lstset{
	numbers=left,
	numberstyle=\tiny,
	numbersep=5pt,
	breaklines=true,
	showstringspaces=false,
	frame=l ,
	xleftmargin=5pt,
	xrightmargin=5pt,
	basicstyle=\ttfamily\small,
	stepnumber=1,
	keywordstyle=\color{blue},          % keyword style
  	commentstyle=\color{dkgreen},       % comment style
  	stringstyle=\color{mauve}         % string literal style
}

%-----------------------------------
%Literaturverzeichnis Einstellungen
%-----------------------------------

% Biblatex

\usepackage{url}
\urlstyle{same}

%%%% DIN 1505 Leitfaden
\usepackage[backend=biber, style=din, maxcitenames=2]{biblatex}%iso date format für YYYY-MM-DD

%% et al. anstatt u. a. bei mehr als drei Autoren.
\DefineBibliographyStrings{ngerman}{ 
	andothers = {{et\,al\adddot}},             
}
\DefineBibliographyStrings{english}{ 
	andothers = {{et\,al\adddot}},             
}

% Bib-Datei einbinden
% Replacement string for Zotero: \n\s+file = \{.*\},
\addbibresource{literatur/literatur.bib}

% Zeilenabstand im Literaturverzeichnis ist Einzeilig
% siehe Leitfaden S. 14
\AtBeginBibliography{\singlespacing}

%-----------------------------------
% Silbentrennung
%-----------------------------------
\usepackage{hyphsubst}
\HyphSubstIfExists{ngerman-x-latest}{%
\HyphSubstLet{ngerman}{ngerman-x-latest}}{}

%-----------------------------------
% Pfad fuer Abbildungen
%-----------------------------------
\graphicspath{{./}{./abbildungen/}}

%-----------------------------------
% Weitere Ebene einfügen
%-----------------------------------
\input{skripte/weitereEbene}

%-----------------------------------
% Zeilenabstand 1,5-zeilig
%-----------------------------------
\usepackage{setspace}
\onehalfspacing

%-----------------------------------
% Absätze durch eine neue Zeile
%-----------------------------------
\setlength{\parindent}{0mm}
\setlength{\parskip}{0.8em plus 0.5em minus 0.3em}

\sloppy					%Abstände variieren
\pagestyle{headings}

%----------------------------------
% Präfix in das Abbildungs- und Tabellenverzeichnis aufnehmen, statt nur der Nummerierung (siehe Issue #206).
%----------------------------------
\KOMAoption{listof}{entryprefix} % Siehe KOMA-Script Doku v3.28 S.153
\BeforeStartingTOC[lof]{\renewcommand*\autodot{:}} % Für den Doppelpunkt hinter Präfix im Abbildungsverzeichnis
\BeforeStartingTOC[lot]{\renewcommand*\autodot{:}} % Für den Doppelpunkt hinter Präfix im Tabellenverzeichnis

%----------------------------------
% Rename list of to index of
%----------------------------------
\addto\captionsenglish{\renewcommand{\listfigurename}{Index of Figures}}
\addto\captionsenglish{\renewcommand{\listtablename}{Index of Tables}}

%-----------------------------------
% Abkürzungsverzeichnis
%-----------------------------------
\usepackage[printonlyused]{acronym}

%-----------------------------------
% Index of Symbols
%-----------------------------------
% Quelle: https://www.namsu.de/Extra/pakete/Listofsymbols.pdf
\usepackage[final]{listofsymbols}

%-----------------------------------
% Index of Formulae
%-----------------------------------
\DeclareCaptionType{mycapequ}[Formula][Index of Formulae]
\DeclareCaptionListFormat{mycapequlist}{Formula~#2:}
\captionsetup[mycapequ]{listformat=mycapequlist}

%-----------------------------------
% Index of Code Listings
%-----------------------------------
\DeclareCaptionType{mycapcode}[Code Listing][Index of Code Listings]
\DeclareCaptionListFormat{mycapcodelist}{Code Listing~#2:}
\captionsetup[mycapcode]{listformat=mycapcodelist}
\newenvironment{codeenv}{}{}

%-----------------------------------
% PDF Meta Daten setzen
%-----------------------------------
\usepackage[hyperfootnotes=false]{hyperref} %hyperfootnotes=false deaktiviert die Verlinkung der Fußnote. Ansonsten inkompaibel zum Paket "footmisc"
% Behebt die falsche Darstellung der Lesezeichen in PDF-Dateien, welche eine Übersetzung besitzen
% siehe Issue 149
\makeatletter
\pdfstringdefDisableCommands{\let\selectlanguage\@gobble}
\makeatother

\hypersetup{
    pdfinfo={
        Title={\myTitel},
        Subject={\myStudiengang},
        Author={\myAutor},
        Build=1.1
    }
}

%-----------------------------------
% Umlaute in Code korrekt darstellen
% siehe auch: https://en.wikibooks.org/wiki/LaTeX/Source_Code_Listings
%-----------------------------------
\lstset{literate=
	{á}{{\'a}}1 {é}{{\'e}}1 {í}{{\'i}}1 {ó}{{\'o}}1 {ú}{{\'u}}1
	{Á}{{\'A}}1 {É}{{\'E}}1 {Í}{{\'I}}1 {Ó}{{\'O}}1 {Ú}{{\'U}}1
	{à}{{\`a}}1 {è}{{\`e}}1 {ì}{{\`i}}1 {ò}{{\`o}}1 {ù}{{\`u}}1
	{À}{{\`A}}1 {È}{{\'E}}1 {Ì}{{\`I}}1 {Ò}{{\`O}}1 {Ù}{{\`U}}1
	{ä}{{\"a}}1 {ë}{{\"e}}1 {ï}{{\"i}}1 {ö}{{\"o}}1 {ü}{{\"u}}1
	{Ä}{{\"A}}1 {Ë}{{\"E}}1 {Ï}{{\"I}}1 {Ö}{{\"O}}1 {Ü}{{\"U}}1
	{â}{{\^a}}1 {ê}{{\^e}}1 {î}{{\^i}}1 {ô}{{\^o}}1 {û}{{\^u}}1
	{Â}{{\^A}}1 {Ê}{{\^E}}1 {Î}{{\^I}}1 {Ô}{{\^O}}1 {Û}{{\^U}}1
	{œ}{{\oe}}1 {Œ}{{\OE}}1 {æ}{{\ae}}1 {Æ}{{\AE}}1 {ß}{{\ss}}1
	{ű}{{\H{u}}}1 {Ű}{{\H{U}}}1 {ő}{{\H{o}}}1 {Ő}{{\H{O}}}1
	{ç}{{\c c}}1 {Ç}{{\c C}}1 {ø}{{\o}}1 {å}{{\r a}}1 {Å}{{\r A}}1
	{€}{{\EUR}}1 {£}{{\pounds}}1 {„}{{\glqq{}}}1
}

%-----------------------------------
% Kopfbereich / Header definieren
%-----------------------------------
\pagestyle{fancy}
\fancyhf{}
% Seitenzahl oben, mittig, mit Strichen beidseits
% \fancyhead[C]{-\ \thepage\ -}

% Seitenzahl oben, mittig, entsprechend Leitfaden ohne Striche beidseits
\fancyhead[C]{\thepage}
%\fancyhead[L]{\leftmark}							% kein Footer vorhanden
% Waagerechte Linie unterhalb des Kopfbereiches anzeigen. Laut Leitfaden ist
% diese Linie nicht erforderlich. Ihre Breite kann daher auf 0pt gesetzt werden.
\renewcommand{\headrulewidth}{0.4pt}
%\renewcommand{\headrulewidth}{0pt}

%-----------------------------------
% Damit die hochgestellten Zahlen auch auf die Fußnote verlinkt sind (siehe Issue 169)
%-----------------------------------
\hypersetup{colorlinks=true, breaklinks=true, linkcolor=darkblack, citecolor=darkblack, menucolor=darkblack, urlcolor=darkblack, linktoc=all, bookmarksnumbered=false, pdfpagemode=UseOutlines, pdftoolbar=true}
\urlstyle{same}%gleiche Schriftart für den Link wie für den Text

%-----------------------------------
% Start the document here:
%-----------------------------------
\begin{document}

\pagenumbering{Roman}								% Seitennumerierung auf römisch umstellen
\newcolumntype{C}{>{\centering\arraybackslash}X}	% Neuer Tabellen-Spalten-Typ:
%Zentriert und umbrechbar

%-----------------------------------
% Textcommands
%-----------------------------------
%----------------------------------
%  TextCommands
%----------------------------------
\renewcommand{\symheadingname}{\langde{Symbolverzeichnis}\langen{Index of Symbols}}
\newcommand{\abbreHeadingName}{\langde{Abkürzungsverzeichnis}\langen{Index of Abbreviations}}
\newcommand{\AppendixName}{\langde{Anhang}\langen{Appendix}}


%-----------------------------------
% Titlepage
%-----------------------------------
\input{kapitel/titelseite}

%-----------------------------------
% Inhaltsverzeichnis
%-----------------------------------
% Um das Tabellen- und Abbbildungsverzeichnis zu de/aktivieren ganz oben in Documentclass schauen
\setcounter{page}{2}
\addtocontents{toc}{\protect\enlargethispage{-20mm}}% Die Zeile sorgt dafür, dass das Inhaltsverzeichnisseite auf die zweite Seite gestreckt wird und somit schick aussieht. Das sollte eigentlich automatisch funktionieren. Wer rausfindet wie, kann das gern ändern.
\setcounter{tocdepth}{4}
\tableofcontents
\newpage

%-----------------------------------
% Abbildungsverzeichnis
%-----------------------------------
\listoffigures
\newpage

%-----------------------------------
% Tabellenverzeichnis
%-----------------------------------
\listoftables
\newpage

%-----------------------------------
% Abkürzungsverzeichnis
%-----------------------------------
\addcontentsline{toc}{section}{\abbreHeadingName}

\section*{\langde{Abkürzungsverzeichnis}\langen{List of Abbreviations}}

\begin{acronym}[WYSIWYG]\itemsep0pt %der Parameter in Klammern sollte die längste Abkürzung sein. Damit wird der Abstand zwischen Abkürzung und Übersetzung festgelegt
  \acro{Beispiel}{Nicht verwendet, taucht nicht im Abkürzungsverzeichnis auf}
\end{acronym}
\newpage

%-----------------------------------
% Index of Symbols
%-----------------------------------
\addcontentsline{toc}{section}{\symheadingname}
%
%
%
%
%
%
%
% Quelle: https://www.namsu.de/Extra/pakete/Listofsymbols.pdf
% Wie ind er Quelle beschrieben führt das Verwenden von Umlauten oder ß zu einem Fehler.
% Hier werden die Symbole definiert in folgender Form:
% \newsym[Beschreibung]{Symbolbefehl}{Symbol}
\opensymdef
\newsym[Aufrechter Buchstabe]{AB}{\text{A}}
\closesymdef

\listofsymbols
\newpage

%-----------------------------------
% Index of Formulae
%-----------------------------------
\listofmycapequs
\newpage

%-----------------------------------
% Index of Formulae
%-----------------------------------
\listofmycapcodes
\newpage

%-----------------------------------
% Seitennummerierung auf arabisch und ab 1 beginnend umstellen
%-----------------------------------
\pagenumbering{arabic}
\setcounter{page}{1}

%-----------------------------------
% Kapitel / Inhalte
%-----------------------------------
% Die Kapitel werden über folgende Datei eingebunden
% Hinzugefügt aufgrund von Issue 167
%-----------------------------------
% Kapitel / Inhalte
%-----------------------------------
\section{Abstract}
Abstract

\newpage
\section{Theory}
\subsection{Full-Text Search}
Commercial database management has long focused on structured data and the industry requirements have matched those of structured storage applications quite well.
The problem is that only a small part of the data stored is completely structured, while most of it is completely unstructured or only semi-structured, in the form of documents, emails, web pages, etc. \parencite[cf.][p. 7]{hamilton_microsoft_2001} Full-text search describes a search technique in which all words of a document or a full-text database are matched with search criteria, whereby not only exact matches but also word reflections and the like can be searched. A full-text database, as opposed to a regular bibliographic database, contains not only metadata but also the complete textual content of books and similar documents. \parencite[cf.][pp. 2-3]{tenopir_full_1990}\\
With large amounts of data, matching every word of all entries is time-consuming and non-performant. To improve this process, a full-text search is divided into an indexing and query phase. In the indexing phase, all words found to be irrelevant, e.g. 'and' or 'the', are ignored by matching them against stoplists, words are normalized, e.g. the capitalization of words, and are merged into an index. \parencite[cf.][p. 11]{coles_pro_2009} In the query phase, full-text query predicates are used to execute search queries. These allow not only a search for exact matches but also generational forms. Generational forms can be, for example, words that stem from the same word or alternative search terms using a language-specific thesaurus. A query processor then calculates the most efficient query plan which delivers the required results. The previously created index is searched for documents and text passages that match the search, and the results are returned in a ranked order. \parencite[cf.][pp. 11-12]{coles_pro_2009}\\
To determine a rank for a search result the quality has to be measured. One symbol is precision \symp{}. \parencite[cf.][pp. 13-15]{coles_pro_2009}
\begin{mycapequ}[H]
    \caption{Calculating precision}
    \begin{tcolorbox}[ams equation]
        p = \frac{n}{d}
    \end{tcolorbox}
    \cite[Source:][p. 14]{coles_pro_2009}
  \end{mycapequ}
\subsubsection{MS SQL Server Search Architecture}
\ac{SQL} Server uses the same access method and infrastructure for full-text search as other \ac{MS} products and the Index Service for file systems. This decision enables standardized semantics for full-text search of data in relational databases, web-hosted data, and data stored in the file system and mail systems. On \ac{SQL} servers, not only simple strings can be indexed, but also data structures, such as \ac{HTML} and \ac{XML}, and even complex documents, such as \ac{PDF}, Word, PowerPoint, Excel and other custom document formats. \parencite[cf.][p. 7]{hamilton_microsoft_2001}\\
The architecture can be divided into five modules, which interact with each other to perform a full-text search. (See Figure \ref{fig:sql_search_architecture})\\
The \textbf{content reader} scans indexed data stored in \ac{SQL} Server tables to assemble data and its associated metadata packets. These packets are then injected into the main search engine, which triggers the search engine filter daemon to consume the data.\\
Depending on the content, the \textbf{filter daemon} calls different filters, which parse the content and output so-called chunks of the processed text. A chunk is a related section with relevant information about this section like the language-id of the text. These chunks are output separately for any properties, which can be elements like the title, an author or other content-specific elements.
\begin{figure}[H]
    \caption{Architecture of MS SQL Server Full-Text Search}
    \label{fig:sql_search_architecture}
    \includegraphics[width=0.9\textwidth]{sql_search_architecture.png}
    \\
    \cite[Source:][p. 8]{hamilton_microsoft_2001}
\end{figure}
\textbf{Word breakers} split the chunks into keywords and additionally provide alternative keywords and the corresponding position in the text. Word breakers can recognize human languages and on \ac{SQL} Server several word breakers for different languages are installed by default. The generated keywords and metadata are passed on to the \ac{MS} Search process, which processes the data with an indexer.\\
The \textbf{indexer} generates an inverted keyword list with a batch containing all keywords of one or more items. These indexes are compressed to use memory efficiently, this may lead to high costs for updates of these indexes. Therefore a stack of indexes is maintained. New documents first create their small indexes, which are regularly merged into a larger index, which in turn is merged into the base index. This stack can be deeper than three, but the concept remains and allows a strongly compressed index without driving the update costs too high. If a keyword is searched, all indexes are accessed, so the depth should still be kept reasonable.\\
A \textbf{query processor} manages the insertion and merge operations and collects statistics on distribution and frequency for ranking purposes and query execution. \parencite[cf.][pp. 8-9]{hamilton_microsoft_2001}
\subsubsection{MS SQL Server Full-Text Query Features}
Full-text indexes can be created on \ac{SQL} Servers with the \ac{DDL} statement \lstinline[language=SQL]$CREATE INDEX$ and can make use of other \ac{SQL} Server utilities; these include backup and restore and attachment of databases. There are three options to create and manage indexes on \ac{SQL} Servers. \textbf{Full Crawl} always rebuilds the whole full-text index by scanning the entire table. \textbf{Incremental Crawl} logs the timestamp of the last re-index and retains changes by storing them in a column. \textbf{Change Tracking} enables a near real-time validity between the full-text index and the table by tracking changes to the indexed data using the \ac{SQL} Server Query Processor. \parencite[cf.][p. 9]{hamilton_microsoft_2001}\\
Full-text search is represented in \ac{SQL} with three possible constructs: \parencite[cf.][p. 9]{hamilton_microsoft_2001}
\begin{enumerate}
    \item Contains Predicate: A contains predicate is true if one of the specified columns contains terms that satisfy the specified search condition. E.g. \lstinline[language=SQL]$Contains(author, ('Ag* or "Marc Miller"'))$ will match entries where the column author contains words like 'Ag', 'Agatha', or 'Marc Miller'.
    \item Freetext Predicate: Freetext predicates are true if one of the specified columns contains terms that stem from the terms in the specified search condition. E.g. \lstinline[language=SQL]$Freetext(content, 'fishing')$ will match entries where content contains words like 'fishing', 'fish', or 'fisher'.
    \item ContainsTable and FreetextTable: ContainsTable and FreetextTable are functions that match entries similar to their corresponding function, but additionally return multiple matches including a ranking for each entry and the entire corpus.
\end{enumerate}
The search conditions of these constructs can be of various types to find the intended results: \parencite[cf.][p. 9]{hamilton_microsoft_2001}
\begin{enumerate}
    \item Keyword, phrase, prefix: E.g. 'fishing', 'Marc Miller', 'Ag*'
    \item Inflections and Thesaurus: E.g. \lstinline[language=SQL]$Contains(*, 'FORMSOF(INFLECTIONAL, fishing) AND FORMSOF(THESAURUS, boat)')$ will find all entries containing words that stem from 'fishing' and all words sharing the meaning with 'boat' (Thesaurus support).
    \item Weighted terms: Keywords and phrases can be assigned a relative weight to impact the rank of entries. E.g. \lstinline[language=SQL]$ContainsTable(*, 'ISABOUT(generator weight (.7), full-text weight (.3))')$ will rank entries higher in the result corpus which mention 'generator' over 'full-text'.
    \item Proximity: E.g. \lstinline[language=SQL]$Contains(*, 'corn NEAR salad')$ contains the proximity term 'NEAR' to match entries where 'corn' appears close to 'salad'.
    \item Composition: E.g. \lstinline[language=SQL]$Contains(*, 'full-text AND NOT database')$ uses two search query components that are composed using a term like 'AND', 'OR', or 'AND NOT'.
\end{enumerate}

\newpage
\section{Implementation}
When using the full-text search, large parts of the SQL statements needed to describe the search are the same, since the search criteria are defined as either WHERE conditions or JOIN criteria. If you want to define a full-text search, you usually use a combination of the given functions. In MSSQL this would be for example CONTAINS or FORMSOF. Therefore I want to develop a query language where you only have to specify this combination of functions and a few parameters to generate the corresponding SQL.
\subsection{Language definition}
The first step to define a language is to define its purpose. In this case, there should be functions that represent full-text functions. Furthermore, one must be able to pass parameters to these functions and one should be able to combine both parameters and functions with logical operators and, or and not.
To announce a function, this query language uses an '@', e.g. '@contains'. From programming languages of the C-family one recognizes the use of parentheses '()' to define parameters. To avoid later confusion with parentheses used for logical grouping, this language uses the colon ':' to enclose parameters. For now, a parameter is defined as a simple word or phrase, which is delimited with quotes '"'. These few rules already allow the definition of a query, such as \lstinline[language=Fulltext-Search]$@contains:apple:$ where 'contains' is the name of a function.
This first set of rules can be written in \ac{EBNF} as:
\begin{grammar}
    <search> ::= '@'<function>':'<parameter>':'; \\
    <function> ::= 'contains'; \\
    <parameter> ::= <word>|' '' '\{[' ']<word>\}' '' '; \\
    <word> ::= \{'a'-'z'|'A'-'Z'\};
\end{grammar}
Note that the function variable only includes 'contains'. In future definitions, it should accept the different functions that are going to be defined.\\
A feature that is also needed is the logical combination and negation of multiple search terms. For example, it should be possible to search for 'apple' or 'tree' and not 'worm'. For and the language accepts the characters '\&' and '+', for or it accepts '|' and for negation it accepts '!' and '-'. To cover all possible logical operations, groups are also needed to allow precedence between the different operators. For this parentheses are used. Using groups it is now possible to build a logic like 'apple' AND NOT('tree' OR 'worm'), where the whole statement inside the parentheses is processed negated, and prioritized instead of being processed from left to right.
\subsection{Lexer}
The first part of a code generator is the lexer. A lexer gets a file or in this case a string as input and divides this input into a series of tokens. So the input \lstinline[language=Fulltext-Search]$@contains:apple:$ becomes the tokens: '@', 'contains', ':', 'apple' and ':'. These tokens are not interpreted yet but are only being recognized as separate characters. To achieve this in code the crate logos is used, to avoid writing redundant code. To understand the code written in lexer.rs what follows is a short explanation of how this crate is used in the context of this prototype.\\
\begin{mycapcode}[H]
    \caption{Basic token defintions}
    \lstinputlisting[language=Rust, linerange={26-28}]{code/code_gen/lexer.rs}
    \vdots
    \lstinputlisting[language=Rust, linerange={38-55}]{code/code_gen/lexer.rs}
    \vdots
    \lstinputlisting[language=Rust, linerange={81-81}]{code/code_gen/lexer.rs}
    \centerline{Source: lexer.rs}
\end{mycapcode}

\input{kapitel/code/code}
\input{kapitel/summary/summary}


%-----------------------------------
% Appendix / Anhang
%-----------------------------------
\newpage
\section*{\AppendixName} %Überschrift "Anhang", ohne Nummerierung
\addcontentsline{toc}{section}{\AppendixName} %Den Anhang ohne Nummer zum Inhaltsverzeichnis hinzufügen

\begin{appendix}
	% Nachfolgende Änderungen erfolgten aufgrund von Issue 163
	\makeatletter
	\renewcommand\@seccntformat[1]{\csname the#1\endcsname:\quad}
	\makeatother
	\addtocontents{toc}{\protect\setcounter{tocdepth}{0}} %
	\renewcommand{\thesection}{\AppendixName\ \arabic{section}}
	\renewcommand\thesubsection{\AppendixName\ \arabic{section}.\arabic{subsection}}
	\section{main.rs}
\lstinputlisting[language=Rust]{code/main.rs}
\section{lexer.rs}
\lstinputlisting[language=Rust]{code/lexer.rs}
\section{parser.rs}
\lstinputlisting[language=Rust]{code/parser.rs}
\section{ast.rs}
\lstinputlisting[language=Rust]{code/ast.rs}
\section{generator.rs}
\lstinputlisting[language=Rust]{code/generator.rs}
\section{mod.rs}
\lstinputlisting[language=Rust]{code/mod.rs}
\section{base.html}
\lstinputlisting[language=HTML]{code/base.html}
\section{search.html}
\lstinputlisting[language=HTML]{code/search.html}
\section{result.html}
\lstinputlisting[language=HTML]{code/result.html}
\end{appendix}
\addtocontents{toc}{\protect\setcounter{tocdepth}{2}}

%-----------------------------------
% Literaturverzeichnis
%-----------------------------------
\newpage
\printbibliography[heading=bibintoc,title={\langde{Literaturverzeichnis}\langen{Bibliography}}]

\input{kapitel/appendix/declaration}
\end{document}
